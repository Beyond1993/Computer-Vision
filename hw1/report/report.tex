%% Latex Template for 16-720J student reports
%% You should maintain the template format wherever practical
%% Gary Overett 2015

\documentclass[12pt]{article}
\usepackage{amsmath}
\usepackage{amssymb}
\usepackage{amsthm}
\usepackage{amscd}
\usepackage{amsfonts}
\usepackage{graphicx}
\usepackage{subcaption}
\usepackage{fancyhdr}
\usepackage{tablefootnote}
\usepackage{subcaption}
\usepackage[table]{xcolor}
\usepackage{array,multirow}
\usepackage{hyperref}
\usepackage{enumerate}
\usepackage{bm}

\topmargin-2cm
\textheight+23cm

\textwidth6.5in

\setlength{\topmargin}{0in} \addtolength{\topmargin}{-\headheight}
\addtolength{\topmargin}{-\headsep}

\setlength{\oddsidemargin}{0in}

\oddsidemargin  0.0in \evensidemargin 0.0in 

\newcounter{list}

\begin{document}

\title{16720J: Homework 1 - Planar Homographies}

\author{Name - YOUR NAME HERE}

\maketitle

\setcounter{section}{2}

\section{Planar Homographies: Theory (40pts)}

\subsection{(10pts)}

Prove that there exists an $\bm{ H}$ that the satisfies homography equation.
\\
The easiest way to show this is to assume ???. Then the points in the plane are of the form $\bm{[?\,?\,?\,?]^T}$

Therefore, the original equations for $\bm{p1}$ and $\bm{p2}$:

\begin{equation}
  p_1 \equiv \left[ \begin{matrix}
      m_{11} & m_{12} & m_{13} & m_{14} \\
      m_{21} & m_{22} & m_{23} & m_{24} \\
      m_{31} & m_{32} & m_{33} & m_{34}
    \end{matrix} \right]_1
\left[ \begin{matrix}
?\\
?\\
?\\
?
\end{matrix} \right]
\end{equation}

\begin{equation}
  p_2 \equiv \left[ \begin{matrix}
      m_{11} & m_{12} & m_{13} & m_{14} \\
      m_{21} & m_{22} & m_{23} & m_{24} \\
      m_{31} & m_{32} & m_{33} & m_{34}
    \end{matrix} \right]_2
\left[ \begin{matrix}
?\\
?\\
?\\
?
\end{matrix} \right]
\end{equation}

can be written as:

\begin{equation}
  \ldots
\end{equation}

\ldots

therefore there exists an $\bm{H}$ where

\begin{equation}
  p_2 \equiv \ldots \equiv Hp1
\end{equation}

When does this fail?

\ldots


\pagebreak

\subsection{(10pts)}
\label{rot}

Prove that there exists an $\bm{H}$ that satisfies homography equation given two cameras separated by a pure rotation.

\begin{equation*}
  p_1 = K_1 [I\,0] P
\end{equation*}
\begin{equation}
  p_2 = K_2 [R\, 0] P
\end{equation}

\ldots

Now you are trying to find that the H exists. It will be a good idea to try to manipulate these expressions in the
direction of some expression

\begin{equation}
  p_2 \equiv \{ \mbox{your expression here} \} p_1
\end{equation}

so

\begin{equation}
  H \equiv \mbox{your clever expression here} \}
\end{equation}

\ldots



\subsection{(5pts)}

From Section \ref{rot}

\begin{equation}
  H \equiv \mbox{your clever expression here} \}
\end{equation}

therefore

\begin{equation}
  H^2 \equiv \ldots
\end{equation}


\subsection{(5pts)}

Why is the planar homography not completely sufficient to map any arbitrary scene image to another viewpoint?

\ldots

{Your thoughts here}



\subsection{(5pts)}

We have a set of points $\bm{p^i_1}$ in an image taken by camera $\bm{C_1}$ and points $\bm{p^i_2}$ in an image taken by
$\bm{C_2}$. Suppose we know there exists an unknown homography $\bm{H}$ such that

\begin{equation}
\bm{p^i_1 \equiv Hp^i_2}
\end{equation}

Assume the points are homogeneous coordinates in the form $\bm{p^i_j = (x^i_j,y^i_j,1)^T}$. For a single point pair, write a matrix
equation of the form
\begin{equation}
  \bm{Ah = 0}
\end{equation}

Where $\bm{h}$ is a vector of the elements of $\bm{H}$ and $\bm{A}$ is a matrix composed of the point coordinates.

\ldots

HINT: we are thinking about the relation

\begin{equation}
\bm{p_1 \equiv Hp_2}
\end{equation}

but we want something in the form $Ax = 0$

Now look at the lecture notes for ``things $=0$''.

Good luck!


\section{Planar Homographies: Implementation (30pts)}


\subsection{(15pts)}

\ldots
See {\tt computeH.m} for template and hints
\ldots

\subsection{(15pts)}

\begin{enumerate}[a)]
\item \ldots
\item See {\tt create\_p1p2.m}
\item See {\tt warp2PNCpark.m}
\item See {\tt q42checker.m} and add the image to this report

  \LaTeX users can add the image with the commented code in {\tt report.tex}

%% \begin{figure}[ht!]
%% \caption{q42checker output.}
%% \centering \includegraphics[width=0.9\linewidth]{path_to_your_image} %% edit here
%% \label{fig:q42checker}
%% \end{figure}

\end{enumerate}

\section{Panoramas (30pts)}

\subsection{(15pts)}

You're doing great! I think you've got this now.

\subsection{(15pts)}

You're doing great! I think you've got this now.

\end{document}
